\chapter{Analyse}
In diesem Kapitel wird die Problemlage im Bezug zum derzeitigen Stand der Technik analysiert.
Etablierte Verfahren werden mithilfe eines geeigneten Testdatensatzes anschließend evaluiert.
Das geeignetste Verfahren wird für den weiteren Verlauf der Arbeit bestimmt.
* Vorarbeit -> Ausarbeiten des bestmöglichen Algorithmus\newline

* Verkehrskameras laden\newline
* Verkehrsbilder laden über Verkehrskameras\newline
* Auswahl verschiedener Ansätze\newline
* Vergleich der Ansätze in Bezug auf Ressourcenlast und Effizienz bzw. Effektivität\newline
* Datensätze generieren, um Vergleichskriterien zu erstellen\newline

\section{Stand der Technik}
Bildverarbeitung ist ein Thema, dass schon sehr lange im Bereich der Informationstechnik und Informatik erforscht wird.
Besonders durch derzeitige Entwicklungen in der künstlichen Intelligenz und der Objekterkennung bekommt das Thema in der heutigen Zeit eine hohe Bedeutung für die technische Entwicklung. Im folgenden Kapitel werden einige der zum derzeitigen Zeitpunkt etablierten Verfahren der Bildverarbeitung im Kontext der Aufgabenstellung evaluiert und bewertet.

\subsection{Statische Verfahren}
\subsubsection{Statische Pixel Anaylse}
\cite{bin2001vehicle}
\subsubsection{Kantenerkennung}
\cite{crouzil2016automatic}
\cite{gupte2000algorithms}
\subsection{Dynamische Verfahren}
\subsubsection{Neuronale Netze}
Neuronale Netze sind ein Beispiel für Verfahren aus dem maschinellen Lernen. 
Die Arbeit \cite{hkkDhbw} geht im Detail darauf ein wie das Problem der Stauerkennung auf Bildern mithilfe dieses Verfahrens gelöst werden kann. 
Jedoch fällt bei Laufzeit eines komplexen neuronalen Netzes auf, dass das Laden und Benutzen der interne Zustandsmaschine einen relativ hohen Hauptspeicher-Bedarf mit sich bringt.
Der Vorteil der sich jedoch hierdurch bietet ist, dass Bilder nicht statisch analysiert werden, sondern das Netz dynamisch auf Bilde und Verhältnisse (Perspektive und Situation) trainiert wird.
Dieses Verfahren wurde für die folgende Implementierung nicht gewählt, da diese Arbeit einen besonderen Fokus auf die Ressourcensparsamkeit des Verfahrens setzt.
* Gewählter Ansatz des letzten Jahrgangs\newline
* Vermutlich bester Ansatz, da keine statische Analyse der Bilder durchgeführt wird\newline
	sondern dynamisch auf konkrete Bilder und Verhältnisse (Perspektive und Situation) trainiert wird
* Nicht durchgeführt, da feststand, dass der Ansatz zu Ressourcenintensiv ist und Ziel der Arbeit war Ressourcen zu sparen\newline

\subsubsection{Background Subtraction}

\section{Bestimmung des Testdatensatzes}
\subsection{Verkehrskameras laden}
* SVZ-BW stellt Kamerabilder für verschiedene Kameras zur Verfügung\newline
* Liste von Verkehrskameras laden und verarbeiten\newline % http://www.svz-bw.de/kamera/kamera_A.txt
* Koordinaten in EPSG:25832 gegeben\newline % http://spatialreference.org/ref/epsg/etrs89-utm-zone-32n/
* Übliche Projektion EPSG:4326\newline % http://spatialreference.org/ref/epsg/wgs-84/
* Umrechnen der Koordinaten in Zielprojektion zur Auswertung der Kamerapositionen\newline

\subsection{Verkehrsbilder laden}
* Über verarbeitete Kameraliste zugriff auf Verkehrskameras\newline
* Zugriff nur möglich mit korrektem Referrer HTTP-Header (www.svz-bw.de)\newline
* Bilder zu gewünschten Kameras regelmäßig abfragen\newline
* SVZ-BW hat timeout und blockt anfragen nach bestimmter Zeit :(\newline

\subsection{Vergleichsdatensätze generieren}
* Zur Auswertung der Ansätze sind Vergleichsdaten nötig\newline
* Viele Bilder zu Verschiedenen Verkehrskameras laden\newline
* Verschiedene Tageszeiten verwenden (morgens, mittags, abends, nachts)\newline
* Verschiedene Witterungsverhältnisse einbeziehen sofern möglich (Nebel, Regen, Sonne, ...)\newline
* Auswertung ob Stau oder kein Stau (bzw. Verkehrslage einschätzen)\newline
* Bilder über einen Ansatz Vorkategorisieren und manuell nachbessern\newline

\section{Auswahl des Verfahrens}
* Augenmerk auf Genauigkeit

\subsection{Helligkeit}
* Histogramm des Bildes bilden\newline
* Grauwertklassen auswerten\newline
* Anhand von Otsu über Schwellwert Klassifikation des Bildes vornehmen\newline
* Benötigt wenig Ressourcen, aber schlechte Performance weil zugrundeliegende Bilder \newline
	natürlich sind und Otsu am besten auf binarisierten Bildern arbeitet (schwarz weiß)
	
\subsection{Haar-Features}
* Erkennen der Autos auf Bildern über Haar-Features und Anschließend zählen, um Stau festzustellen\newline
* OpenCV hat Features für Autos vorgegeben\newline
* Erkennt Autos nicht immer verlässlich\newline
* Arbeitet besser auf Bildreihen (Videos) um Bewegung der speziellen Autos zur verfolgen\newline
* Verkehrskameras liefern über die Zeit zwar viele Bilder, aber mit zu großem zeitlichen Abstand\newline
* Einzelne Autos also nicht verfolgbar\newline
* Schlechte Resultate zur Stauerkennung\newline

\subsection{Edge detection}
* Erkennen der Kanten über Canny\newline
* Anhand der Kanten Bild Segmentieren und Konturen herausarbeiten zur Erkennung von Merkmalen und Klassifikation\newline
* Auflösung der Kamerabilder zu gering und Bilder recht unscharf mit viel zu Rauschen\newline
* Zu viele Kanten werden erkannt, um Autos verlässlich erkennen zu können\newline
* Mittelung des Bildes über Gauß zeigt keine signifikante Verbesserung aufgrund der zu niedrigen Qualität\newline
	
\subsection{Background Subtraction}
* Viele Bilder über möglichst kurzen Zeitabstand aufsummieren\newline
* Anhanddessen Hintegrund errechnen\newline
* Hintegrund vom Zielbild abziehen\newline
* Übrig bleiben nur noch Autos\newline
* Morphologische Operatoren zur Nachoptimierung\newline
* Konturen zählen -> Anzahl der Autos\newline
* Über Anzahl der Autos Rückschlüsse über Verkehrssituation ziehen\newline
* Verschiedene BGS algorithmen in OpenCV verfügbar (GSOC, KNN, CNT, GMG, LSBP, MOG, MOG2)\newline
* MOG liefert beste Ergebnisse für konkretes Einsatzgebiet\newline
* Jedoch mehrere Bilder nötig, um Hintergrund verlässlich zu erkennen\newline
* Bei absolutem Stau (keine Bewegung der Autos über längere Zeit ~10min) keine Erkennung des Hintergrundes möglich\newline

\subsection{Auswahl}
* BGS is top
