\chapter{Grundlagen}

Dieses Kapitel beschreibt Grundlagen, die für das weitere Verständnis der Arbeit benötigt werden.

\section{Straßenverkehrszentrale Baden-Württemberg} % Mi
Aktuelle Verkehrsinformationen auf den Autobahnen und Bundesstraßen in Baden-Württemberg und der näheren Umgebung lassen sich über die Domain der Straßenverkehrszentrum Baden-Württemberg abrufen. Die Straßenverkehrszentrale nimmt sich dabei unter Anderem zum Ziel, Stau auf den Straßen zu reduzieren und Autofahrer über aktuelle Verkehrsbehinderungen zu informieren \cite{svzbw}.

\section{Histogramm} % Mi

\section{Faltungskerne} % Mo
Faltungskerne beschreiben Filter in der Bildverarbeitung, die über die diskrete Faltung im 2-dimensionalen
Raum auf ein Bild angewendet werden.


Grundsätzlich ist die 1-dimensionale Faltung im kontinuierlichen Raum definiert durch die Integration zweier Funktionen {\em g} und {\em f} an einem Punkt {\em t}, wobei die Funktion {\em g} gespiegelt wird, also auf {\em f} gefaltet wird:

$$ (f * g)(t) = \int_{-\infty}^{\infty} f(\tau)g(t - \tau) d\tau $$

Für die Bildverarbeitung ist jedoch die Faltung im kontinuierlichen 2-dimensionalen Raum relevant.
Hierfür wird statt dem Integral, die Doppelsumme über alle Werte {\em n} gebildet (in {\em x}- und {\em y}-Richtung) und der Filter {\em k} mit dem Bild {\em I} an einem Punkt {\em (x, y)} gefaltet:

$$ I\mbox{*}(x, y) = \sum_{i=1}^{n}\sum_{j=1}^{n} I(x - i, y - j)k(i, j) $$

Hiermit wird nun eine Faltungsmatrix, bzw. ein Faltungskern auf jeden Pixel im Bild angewendet:

$$ k = \left( \begin{array}{rrr}
1 & 1 & 1 \\
1 & 1 & 1 \\
1 & 1 & 1 \\
\end{array}\right) $$

Faltungskerne sind lokale Operatoren, die die Neuberechnung eines Pixels mittels eines Teilbereichs des Bildes durchführen.
Mit solchen Filtern lassen sich Bilder beispielsweise Schärfen, Glätten. Es lassen sich jedoch auch Kanten finden oder Rauschanteile reduzieren.

\subsection{Gauß-Filter}
Der Gauß-Filter ist ein Faltungskern, welcher über eine gaußsche Glockenkurve gebildet wird.
Mit solch einem Filter lassen sich Bilder glätten und somit Rauschanteile reduzieren.
Bilder wirken dadurch weicher, bzw. verwaschen.

Ein möglicher Faltungskern sähe so aus:

$$ \frac{1}{16} \left( \begin{array}{rrr}
1 & 2 & 1 \\
2 & 4 & 2 \\
1 & 2 & 1 \\
\end{array}\right) $$

Wendet man diesen Filter auf ein Bild mit der diskreten Faltung an, erhält man dieses Ergebnis:

\begin{figure}[ht]
   \centering
     \includegraphics[width=11cm]{Bilder/Gaussian_Blur} \\
 \caption{Anwendung eines Weichzeichnungsfilters}
 \source{https://de.wikipedia.org/wiki/Datei:Halftone,_Gaussian_Blur.jpg}{10.3.2019}
 \label{fig:Blur}
\end{figure}

\subsection{Laplace-Filter}
Ein Laplace-Filter ist ein Faltungskern zur Kantendetektion innerhalb eines Bildes.
Unter einer Kante versteht man hierbei eine rasche Veränderung der Helligkeitswerte entlang einer Richtung.

Um Kanten zu finden wird ein Operator auf das Bild angewendet, welcher die zweite Ableitung bildet (Laplace-Operator). Abrupte Schwankungen der Intensitätswerte werden dadurch als Nulldurchgänge sichtbar.

Über die Faltung des Operators der Vorwärtsdiffenz {\em (1 -1)} mit sich selbst, lässt sich ein 1-dimensionaler Faltungskern der zweiten Ableitung bilden: {\em (1 -2 1)}.
Dieser Kern lässt sich transponieren, um ein Bild nicht nur in x-, sondern auch in y-Richtung abzuleiten.
Beide Kerne kombiniert ergeben den Laplace-Filter:

$$ \left( \begin{array}{rrr}
0 & 1 & 0 \\
1 & -4 & 1 \\
0 & 1 & 0 \\
\end{array}\right) $$

Wendet man diesen Filter auf ein Bild an, erhält man folgendes Ergebnis:

\begin{figure}[ht]
   \centering
     \includegraphics[width=11cm]{Bilder/Laplace} \\
 \caption{Anwendung eines Laplace-Filters}
 \source{https://de.wikipedia.org/wiki/Datei:Laplace_beispiel.png}{10.3.2019}
 \label{fig:Laplace}
\end{figure}

Aufgrund eines hohen Rauschanteils in natürlichen Bildern liefert dieser Filter jedoch nicht immer gute Resultate.

\subsection{Sobel-Operator}
Um auf natürlichen Bildern Kanten zuverlässig zu erkennen, kombiniert der Sobel-Operator die Ideen des Gauß- und Laplace-Filters.
Hierbei wird das Bild in eine Richtung über die zentrale Differenz {\em (1 0 -1)} abgeleitet, andere Richtung jedoch über einen Gauß-Filter {\em (1 2 1)} geglättet,
um Rauschanteile zu reduzieren.
Kombiniert man beide Filter miteinander, erhält man den Sobel-Operator

$$ \left( \begin{array}{rrr}
-1 & 0 & 1 \\
-2 & 0 & 2 \\
-1 & 0 & 1 \\
\end{array}\right) $$

Dadurch werden Kanten jedoch nur in eine Richtung erkannt.
Um Kanten in die jeweils andere Richtung zu erkennen, kann die Faltungsmatrix transponiert werden und ebenfalls auf
das Ausgangsbild angewendet werden.

Die beiden erhaltenen Ergebnisse können nun vereint werden, um alle Kanten innerhalb eines Bildes zu erhalten:

\begin{figure}[ht]
   \centering
     \includegraphics[width=11cm]{Bilder/Sobel} \\
 \caption{Anwendung eines Sobel-Operators}
 \source{https://en.wikipedia.org/wiki/File:Valve_sobel_(3).PNG}{10.3.2019}
 \label{fig:Sobel}
\end{figure}

\section{Canny-Edge-Detection} % Mi
\section{Otsu} % Mi
\section{Haar-Features} % Mi

\section{Morphologische Operatoren} % Mo
Morphologische Operatoren werden in der Bildverarbeitung eingesetzt, um die Form von Strukturen innerhalb eines Bildes zu verändern.
Solche Operatoren können sowohl auf Binär-, als auch auf Grauwertbildern angewendet werden, wobei hier lediglich auf die Anwendung bei Binärbildern
eingegangen wird.

Zunächst gibt es einfache Basisoperationen wie Erosion und Dilatation, welche die Basis für weitere, komplexere morphologische Operatoren wie beispielsweise Opening und Closing bilden.

\subsection{Erosion}
Bei der Erosion werden Strukturen, wie der Name vermuten lässt, erodiert bzw. abgetragen.
Hierfür wird zunächst eine Maske definiert. Üblicherweise ein Raster von 3x3 Pixeln, wobei der Kern der Maske auch der Mittelpunkt des Rasters ist.

Für jeden gesetzten Pixel (mit Wert 1) innerhalb des Binärbilds wird nun die Maske aufgelegt. Sind dabei nicht alle anderen Pixel innerhalb der Maske ebenfalls gesetzt, so wird das Pixel im resultierenden Bild auf den Wert 0 gesetzt.
Haben jedoch alle Nachbarn eines Pixels innerhalb der Maske ebenfalls den Wert 1, so verändert sich das Pixel nicht.

Formal wird die Erosion folgendermaßen notiert: $ I \ominus M $. Wobei das Bild {\em I} mit der Maske {\em M} erodiert wird.

Der daraus resultierende Effekt ist, wie in Abbildung~\ref{fig:Erosion} ersichtlich, dass Ränder von Strukturen innerhalb des Bildes abgetragen werden und diese somit verdünnt werden. Es können aber auch Löcher innerhalb der Strukturen vergrößert werden.

\begin{figure}[ht]
   \centering
     \includegraphics[width=11cm]{Bilder/MorphologicalErosion} \\
 \caption{Anwendung der Erosion auf ein Bild}
 \source{https://de.wikipedia.org/wiki/Datei:MorphologicalErosion.png}{12.3.2019}
 \label{fig:Erosion}
\end{figure}

\subsection{Dilatation}
Die Dilatation bildet das Gegenstück zur Erosion. Statt die jeweiligen Strukturen abzutragen, werden diese wachsen gelassen.
Dabei kann die selbe Maske wie beider Erosion verwendet werden. Diese muss ebenfalls für jeden Pixel auf das Binärbild aufgelegt werden. Ist der Kern der Maske, also das Pixel auf welches die Maske gelegt wurde gesetzt, so werden im resultierenden Bild alle Pixel innerhalb der Maske ebenfalls gesetzt. Hat der Kern jedoch den
Wert 0, so verändert sich für dieses Pixel nichts.

Die formale Notation ist $ I \oplus M $, wobei hier Bild {\em I} mit Maske {\em M} dilatiert wird.

Wie in Abbildung~\ref{fig:Dilation} erkennbar, wachsen die Strukturen innerhalb des Bildes. Dabei können vorher getrennte Strukturen zusammenwachsen oder Löcher innerhalb einer Struktur geschlossen werden.

\begin{figure}[ht]
   \centering
     \includegraphics[width=11cm]{Bilder/MorphologicalDilation} \\
 \caption{Anwendung der Dilatation auf ein Bild}
 \source{https://de.wikipedia.org/wiki/Datei:MorphologicalDilation.png}{12.3.2019}
 \label{fig:Dilation}
\end{figure}

\subsection{Opening}
Im Gegensatz zu den vorhergehenden Operationen ist das Opening keine Basisoperation mehr, sondern setzt sich aus den Basisoperationen zusammen.
Dabei wird ein Bild {\em I} zunächst mit einer Maske {\em M} erodiert und anschließend dilatiert:

$$ I \circ M = ( I \ominus M ) \oplus M $$

Die Struktur von hinreichend großen Elementen wird durch diese Operation nicht verändert. Es werden lediglich kleinere Artefakte, welche zumeist durch Störungen oder Rauschen innerhalb des Ursprungsbildes entstanden sind, entfernt, da diese durch die Erosion entfernt werden, aber durch die anschließende Dilatation nicht mehr erzeugt werden können. Es können auch Strukturen, die über eine Verbindungsstelle, welche schmäler als die Maske ist, voneinander getrennt werden, sodass aus einer Struktur zwei getrennte entstehen.
Ein Beispiel lässt sich in Abbildung~\ref{fig:Opening} sehen.

\begin{figure}[ht]
   \centering
     \includegraphics[width=11cm]{Bilder/MorphologicalOpening} \\
 \caption{Anwendung von Opening auf ein Bild}
 \source{https://de.wikipedia.org/wiki/Datei:MorphologicalOpening.png}{12.3.2019}
 \label{fig:Opening}
\end{figure}

\subsection{Closing}
Closing bildet das Gegenstück zu Opening.
Ein Bild {\em I} wird mit einer Maske {\em M} dilatiert und anschließend erodiert:

$$ I \bullet M = ( I \oplus M ) \ominus M $$

Genutzt werden kann dieser Operator, um beispielsweise, wie der Name erschließen lässt, Löcher in Strukturen zu schließen, oder Verbindungen zwischen vorher getrennten Strukturen aufzubauen.
Durch die Dilatation werden Lücken geschlossen, bzw. nahe beieinander liegende Strukturen verbunden. Die Größe der Elemente wird durch die anschließende Erosion wieder zurückgesetzt. Entstandene Verbindungen und geschlossene Lücken bleiben dadurch jedoch erhalten.
Ein Beispiel ist in Abbildung~\ref{fig:Closing} zu sehen.

\begin{figure}[ht]
   \centering
     \includegraphics[width=11cm]{Bilder/MorphologicalClosing} \\
 \caption{Anwendung von Closing auf ein Bild}
 \source{https://de.wikipedia.org/wiki/Datei:MorphologicalClosing.png}{12.3.2019}
 \label{fig:Closing}
\end{figure}

\section{OpenCV} % Mi
