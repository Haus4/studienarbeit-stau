\chapter{Zusammenfassung}

% Kritische, inhaltliche Reflexion von Theorie und Praxis 
\section{Fazit}
Das im Rahmen der Arbeit entwickelte System erfüllt die gestellten Anforderungen an die Ressourceneffizienz.
Dennoch bedeutet eine Neuentwicklung auch Veränderungen im Vergleich zum alten System, welche nicht zwingend die Benutzbarkeit des Systems fördern.

Ein Beispiel hierfür ist der neue optimierte Algorithmus zur Bildauswertung, welcher durchaus weniger Ressourcen verbraucht, aber auch etwas ungenauer als der alte Algorithmus ist.
So musste während der Auswahl des Algorithmus auch ein Ausgleich zwischen Effizienz und Genauigkeit geschaffen werden.
Weiterhin kann das System durch den optimierten Algorithmus auch weniger dynamisch auf Veränderungen des Wetters und der Umgebung reagieren.

Der Vorteil, der sich hierdurch bietet ist der, dass die Implementierung des Algorithmus auch auf den Betrieb im zukünftigen Host-System ausgerichtet ist.
So soll das Backend das System zukünftig auch auf einem geteilten Webhost der Firma 1\&1 laufen können.
Wie bereits in Abschnitt~\ref{sec:infrastructure} angesprochen erfüllt die Infrastruktur des neuen Systems die Anforderungen für den Betrieb in einem geteilten Webhost.

Schlussendlich gab es auch Veränderungen am Frontend des Systems.
So wurde die Benutzerschnittstelle vereinfacht und es gibt eine Audio-Ausgabe über Bluetooth, welche eine Unterstützung von Freisprechanlagen im Auto bietet.
Benutzer des Systems können somit einfach die App starten und sich mit der Freisprechanlage des Autos verbinden, um die komplette Funktionalität des Systems ausnutzen zu können.
\newpage

\section{Ausblick}
Die Implementierung des neuen Systems bietet weiterhin Freiraum für neue Verbesserungen und Features, welche während der Arbeit nicht implementiert werden konnten.

So könnte man auch die Genauigkeit des Systems durch ein optimiertes Background-Modell für den Background-Subtraction Algorithmus verbessern.
Eine weitere Möglichkeit die Genauigkeit des Systems zu Verbessern bietet sich durch das Einfügen weiterer Vor- beziehungsweise Nachbereitungsschritten während der Auswertung von Verkehrskamera-Bildern.
Zusätzlich könnte der Schwellwert für die Trennung zwischen stockendem und fließendem Verkehr durch einen Algorithmus dynamisch bestimmt werden.

Die Straßenverkehrszentrale bietet nicht immer eine zuverlässige Auskunft für Verkehrskamerabilder.
Es könnte daher auch vorkommen, dass die Server für Kamerabilder der Straßenverkehrszentrale für mehrere Tage nicht erreichbar sind.
Um auch in Zukunft eine Ausfallfreiheit des Systems zu garantieren, würde es daher Sinn machen, eine alternative Datenquelle zu finden.

Zusätzlich lässt sich die Benutzerfreundlichkeit des Frontends verbessern, indem man Verbesserungen wie die Erkennung der Entfernung zu Ausfahrten einführen würde.
Weiterhin würde ein Knopf auf der Benutzeroberfläche, welcher die letzte Audio-Ausgabe wiederholt, die Benutzerfreundlichkeit fördern.

Schlussendlich muss die Implementierung des System erstmals ausgiebig auf dem geteilten Webhost getestet werden, bevor neue Verbesserungen eingeführt werden sollten.
Somit wird sichergestellt, dass das System auch weiterhin alle Anforderungen erfüllt und korrekt ist.
