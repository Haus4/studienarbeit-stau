\chapter{Bewertung}
In diesem Kapitel wird das im vorherigen Kapitel beschriebene System beurteilt und mit dem vorhergehenden System verglichen.
Aufgrund der sich unterscheidenden Infrastrukturen der beiden Systeme, ist es nicht möglich einen direkten Vergleich durchzuführen.
Stattdessen müssen verschiedene Kennzahlen bezüglich der zu erfüllenden Anforderungen gefunden werden, um einen Vergleich zwischen den Kennzahlen des jeweiligen Systems durchzuführen.

\section{Abweichungen der Systeme}
Das in dieser Arbeit beschriebene System wurde besonders im Hinblick auf Ressourceneffizienz optimiert.
Somit gibt es auch starke Unterscheidungen zum alten System, welches lediglich ergebnisorientiert entworfen wurde und daher einen erhöhten Ressourcenverbrauch besitzt.

Um weniger Ressourcen zu verbrauchen wurde zum einen auch die Auswertung von Verkehrskamerabildern vereinfacht.
Statt der Verwendung eines neuronalen Netzes, welches einen sehr hohen Ressourcenbedarf mit sich zieht, wird nun ein weniger komplexer Background Subtraction Algorithmus verwendet.

Die Auswertung wird dadurch so stark vereinigt, dass sie sich auch auf die Clients auslagern lässt.
Das Backend kann also als einfacher Zwischenspeicher für Daten der Straßenverkehrszentrale Baden-Württemberg verwendet werden und ist somit auch für den Nutzen auf einem Shared Webhost optimiert.

Durch die Auslagerung der Auswertung auf die Clients ist jedoch auch nun der Großteil der Berechnung auf dem Client zu finden.
Es ist daher wichtig bestimmte Kenngrößen zu finden, um das eher client-lastige neue System mit dem server-lastigen Altsystem zu vergleichen.

*Auswertung Client vs Auswertung Server\newline
*Optimiert für Nutzen auf Shared Webhost\newline

\section{Kennzahlen}
*Serverlast\newline
*Clientlast\newline
*Genauigkeit\newline
*Energieverbrauch\newline
*...\newline

\section{Datenerhebung}
*Wie sind Metriken messbar? Task-Manager, ...\newline
*Wie wurden Daten aufgenommen\newline

\section{Auswertung}
*Darstellung ggf. über Diagramme\newline
*Weniger genau aber auch weniger Ressourcen\newline
