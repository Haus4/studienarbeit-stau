\chapter{Implementierung}
Im folgenden Kapitel wird die praktische Implementierung der in der Arbeit erarbeiteten Methode beschrieben. Die Implementierung umfasst dabei die Schnittstelle zum Nutzer, sowie die Erhebung und Verarbeitung von relevanten Daten.
\section{Backend}
Für die Implementierung der Arbeit wurde ein Backend konzipiert, welches Informationen und Bilder von Verkehrskameras für Clients vorenthält. Das Ziel ist es dabei Daten einem für den Client verständlichen Format aufzubereiten und anschließend möglich effizient zu persistieren. Somit wird ermöglicht, dass der Client auch auf ältere Datensätze der Straßenverkehrszentrale zugreifen kann. Weiterhin ermöglicht das Backend auch das Senden von mehreren Datensätzen innerhalb einer Anfrage. 
\subsection{Infrastruktur}
Im Rahmen der Implementierung des Backends wurden diverse Anforderungen an die zu verwendenden Technologien beachtet. So soll die resultierende Anwendung auf dem Apache Server eines geteilten Webhosts laufen. Der Apache Server des Webhosts erlaubt das Ausführen von PHP-Skripten, welche für Anfragen und Verarbeitung von Daten der Straßenverkehrszentrale Baden-Württemberg verwendet werden können. Für die Persistenz von Daten besitzt der Webhost alternativ zum Dateisystem auch eine MySQL-Datenbank.
--Bild Infraturktur?--

Auf Abbildung X ist zu sehen, dass die Verbindung zum Client über das HTTP-Protokoll verläuft. Der Vorteil ist hierbei, dass das Protokoll bereits von dem Apache Server unterstützt wird und eine Verschlüsselung über TLS eingesetzt werden kann.
\subsection{Datenerhebung}
Die Erhebung von benötigten Daten erfolgt durch HTTP-Anfragen an den Server der Straßenverkehrszentrale. Informationen über die jeweiligen Kameras werden über den SVZ-Server in verschieden Formaten bereitgestellt. So sind die Metadaten der Kameras als Textdatei abgespeichert, während die Bilder direkt im JPEG-Format abgerufen werden können. Unter Metadaten der Verkehrskameras sind bei folgende Informationen zu verstehen: 
\begin{itemize}
\item{Eine eindeutige alphanumerische Kamera-Kennung}
\item{Die entsprechende Autobahnkennung}
\item{Name der nächsten Ausfahrt2 mit Autobahn Kennung}
\item{Position der Kamera}
\end{itemize}
Die Metadaten der Kameras müssen nur einmalig abgerufen und verarbeitet werden, während die Bilder der Verkehrskameras alle 30 Sekunden erneuert werden. Um immer aktuelle Datensätze zu empfangen werden Bilder im 30-Sekunden-Takt angefragt.
\subsection{Datenverarbeitung}
\subsection{Datenpersistenz}

\section{Client}
Der Client wird als eine Android Anwendung realisiert, die durch Kommunikation mit dem Backend und der Straßenverkehrszentrale das Problem der Stauerkennung so effizient und ressourcensparend wie möglich löst.
Durch das Verlagern der Berechnungen auf das mobile Endgerät bleibt die Last des Backends gering und die Rechenleistung des Smartphones wird soweit ausgeschöpft, dass Berechnungen effizient, aber sparsam durchgeführt werden können.

\subsection{OpenCV}
Der Kern der Verarbeitung wird durch Algorithmen der OpenCV Bibliothek~\ref{sec:OpenCV} umgesetzt.
Da OpenCV primär als native Bibliothek zur Verfügung steht, bzw. die für Android nutzbare Java-Schnittstelle auf die native (in C++ geschriebene) Implementierung
zurückgreift, muss OpenCV für die jeweiligen Prozessoren der Android Geräte kompiliert werden, auf denen die Anwendung genutzt werden soll.

Eigentlich bietet OpenCV selbst bereits vorkompilierte Versionen der Bibliothek für alle gängigen CPUs und Android Geräte an, jedoch sind diese nicht vollständig.
OpenCV besteht nämlich eigentlich aus zwei teilen: der Kernbibliothek, also OpenCV selbst, aber zusätzlich noch einem Erweiterungsmodul namens {\em OpenCV-Contrib}.

Diese Erweiterung bietet zusätzlich zu den Standartalgorithmen des Kerns Erweiterungen und diverse andere Algorithmen an, welche Randprobleme abdecken und bei der regulären Nutzung der Bibliothek nicht benötigt werden.

Da für die Umsetzung der Anwendung ein spezieller Background-Subtraction Algorithmus benötigt wird, welcher nur in dem Erweiterungsmodul zur Verfügung steht, wird dieses hierfür benötigt.

Die vorkompilierte Version von OpenCV beinhaltet jedoch nicht die Erweiterungen, sodass diese nicht nutzbar ist.

Um die Bibliothek jedoch nicht selbst für jede Architektur kompilieren zu müssen, gibt es die Möglichkeit die Bibliothek von dritten zu beziehen.
Entwickler unter dem Namen {\em QuickBird Studios} stellen auf der Plattform {\em GitHub} (\url{https://github.com/quickbirdstudios/opencv-android} eine aktuelle Version von OpenCV mit allen Erweiterungen zur Verfüngung, welche ganz einfach in die App eingebunden und genutzt werden kann.

\subsection{Kameras Laden}
\subsection{Geolokalisierung}
\subsection{Richtungsfestellung}
\subsection{Text-To-Speech}
\subsection{Bluetooth}
* SCO Headset ausgabe
* Autoverbindung mit Auto
* Service 